\documentclass[10pt]{article}\usepackage[]{graphicx}\usepackage[]{color}
%% maxwidth is the original width if it is less than linewidth
%% otherwise use linewidth (to make sure the graphics do not exceed the margin)
\makeatletter
\def\maxwidth{ %
  \ifdim\Gin@nat@width>\linewidth
    \linewidth
  \else
    \Gin@nat@width
  \fi
}
\makeatother

\definecolor{fgcolor}{rgb}{0.345, 0.345, 0.345}
\newcommand{\hlnum}[1]{\textcolor[rgb]{0.686,0.059,0.569}{#1}}%
\newcommand{\hlstr}[1]{\textcolor[rgb]{0.192,0.494,0.8}{#1}}%
\newcommand{\hlcom}[1]{\textcolor[rgb]{0.678,0.584,0.686}{\textit{#1}}}%
\newcommand{\hlopt}[1]{\textcolor[rgb]{0,0,0}{#1}}%
\newcommand{\hlstd}[1]{\textcolor[rgb]{0.345,0.345,0.345}{#1}}%
\newcommand{\hlkwa}[1]{\textcolor[rgb]{0.161,0.373,0.58}{\textbf{#1}}}%
\newcommand{\hlkwb}[1]{\textcolor[rgb]{0.69,0.353,0.396}{#1}}%
\newcommand{\hlkwc}[1]{\textcolor[rgb]{0.333,0.667,0.333}{#1}}%
\newcommand{\hlkwd}[1]{\textcolor[rgb]{0.737,0.353,0.396}{\textbf{#1}}}%

\usepackage{framed}
\makeatletter
\newenvironment{kframe}{%
 \def\at@end@of@kframe{}%
 \ifinner\ifhmode%
  \def\at@end@of@kframe{\end{minipage}}%
  \begin{minipage}{\columnwidth}%
 \fi\fi%
 \def\FrameCommand##1{\hskip\@totalleftmargin \hskip-\fboxsep
 \colorbox{shadecolor}{##1}\hskip-\fboxsep
     % There is no \\@totalrightmargin, so:
     \hskip-\linewidth \hskip-\@totalleftmargin \hskip\columnwidth}%
 \MakeFramed {\advance\hsize-\width
   \@totalleftmargin\z@ \linewidth\hsize
   \@setminipage}}%
 {\par\unskip\endMakeFramed%
 \at@end@of@kframe}
\makeatother

\definecolor{shadecolor}{rgb}{.97, .97, .97}
\definecolor{messagecolor}{rgb}{0, 0, 0}
\definecolor{warningcolor}{rgb}{1, 0, 1}
\definecolor{errorcolor}{rgb}{1, 0, 0}
\newenvironment{knitrout}{}{} % an empty environment to be redefined in TeX

\usepackage{alltt}


%\VignetteEngine{knitr::knitr}
%\VignetteIndexEntry{Scalable estimation of hierarchical models}

\usepackage[T1]{fontenc}
\usepackage[utf8]{inputenc}
\usepackage{mathpazo}
\usepackage{amsmath}
\usepackage{amsfonts}
\usepackage[margin=1in]{geometry}
\usepackage{placeins} %% for \FloatBarrier
\usepackage{subcaption}

\usepackage{algorithm}
\usepackage{algorithmic}

\usepackage{parskip}
\usepackage{setspace}
\linespread{1.1}

\usepackage[dvipsnames,svgnames,x11names,hyperref]{xcolor}
\usepackage[colorlinks=true, urlcolor=NavyBlue]{hyperref}

%%make knitr environment line spacing separate from LaTeX
\renewenvironment{knitrout}{\begin{singlespace}}{\end{singlespace}}

\DeclareMathOperator\logit{logit}
\DeclareMathOperator\Chol{Chol }
\DeclareMathOperator\cov{cov}

\newcommand{\pkg}[1]{\emph{#1}}
\newcommand{\proglang}[1]{\textsf{#1}}
\newcommand{\func}[1]{\texttt{#1}}
\newcommand{\class}[1]{\texttt{#1}}
\newcommand{\funcarg}[1]{\texttt{#1}}
\newcommand{\filename}[1]{\textit{#1}}
\newcommand{\code}[1]{\texttt{#1}}
\newcommand{\variable}[1]{\texttt{#1}}
\newcommand{\method}[1]{\textsl{#1}}

 \newcommand{\Prior}{\pi(\theta)}
 \newcommand{\Post}{\pi(\theta|y)}
 \newcommand{\Gtheta}{g(\theta)}
 \newcommand{\Phitheta}{\Phi(\theta|y)}
 \newcommand{\Ly}{\mathcal{L}(y)}
 \newcommand{\Dy}{\mathcal{D}(\theta,y)}


\usepackage[style=authoryear,%
			backend=biber,%
			maxbibnames=99,%
                        bibencoding=utf8,%
                        maxcitenames=1,
                        citetracker=true,
                        dashed=false,
                        maxalphanames=1,%
                        backref=false,%
			doi=true,%
                        url=true, %
			isbn=false,%
                        mergedate=basic,%
                        dateabbrev=true,%
                        natbib=true,%
                        uniquename=false,%
                        uniquelist=false,%
                        useprefix=true,%
                        firstinits=false
			]{biblatex}

 \addbibresource{../../vignettes/bayesGDS.bib}

 \setlength{\bibitemsep}{1em}
\AtEveryCitekey{\ifciteseen{}{\defcounter{maxnames}{2}}}
\DeclareFieldFormat[article,incollection,unpublished]{title}{#1} %No quotes for article titles
\DeclareFieldFormat[thesis]{title}{\mkbibemph{#1}} % Theses like book
                                % titles
\DeclareFieldFormat{pages}{#1} %% no pp prefix before page numbers
\renewbibmacro{in:}{%
  \ifentrytype{article}{}{ % Don't print In: for journal articles
  \printtext{\bibstring{in}\intitlepunct}} %% but use elsewhere
}
\renewbibmacro*{volume+number+eid}{%
  \printfield{volume}%
  \printfield{number}%
  \setunit{\addcomma\addspace}%
  \printfield{eid}}

\DeclareFieldFormat[article]{volume}{\mkbibbold{#1}}
\DeclareFieldFormat[article]{number}{\mkbibparens{#1}}
\DeclareFieldFormat[article]{date}{#1}
\AtEveryBibitem{\clearfield{day}}

\renewbibmacro*{issue+date}{% print month only
  \printtext{%
    \printfield{issue}\addspace%
    \newunit%
%\printtext{\printfield{month}}%
}
  \newunit}

\renewbibmacro*{publisher+location+date}{% no print year
  \printlist{location}%
  \iflistundef{publisher}
    {\setunit*{\addcomma\space}}
    {\setunit*{\addcolon\space}}%
  \printlist{publisher}%
  \setunit*{\addcomma\space}%
%%\printdate
  \newunit}

\renewcommand*{\nameyeardelim}{~} %no comma in cite
\renewcommand{\bibitemsep}{1ex}
\def\bibfont{\small}

\title{Estimating Bayesian Hierarchical Models using \pkg{bayesGDS}}
\author{Michael Braun\\Cox School of Business\\Southern Methodist University}
\date{March 26, 2015}
\IfFileExists{upquote.sty}{\usepackage{upquote}}{}
\begin{document}

\maketitle




\citet{BraunDamien2015}, henceforth known as BD, introduce an
alternative to MCMC for sampling from posterior distributions. The
main advantages of BD over MCMC are
that samples can be collected in parallel, and that the algorithm
is scalable (linear complexity) for hierarchical models with conditionally independent
heterogeneous units.  These features make BD an attractive estimation method
for Bayesian hierarchical models of large datasets.

For the purposes of this vignette, we assume that the reader is familiar with
the BD algorithm, and in particular, Algorithm 1.  In the pages that
follow, we discuss some of the practical issues involved in running the BD
algorithm. The focus is on a few specific aspects of the algorithm:
\begin{enumerate}
\item exploiting the sparsity of the Hessian of the log
  posterior density;
\item finding the posterior mode through nonlinear optimization;
 \item sampling from, and computing the log density of, a
   high-dimensional multivariate normal (MVN) distribution; and
 \item computing the log marginal likelihood of the data under the model.
\end{enumerate}

A common reaction to the BD algorithm, and especially to the need to find
the posterior mode, is ``easier said than done.''  We suspect that this reaction comes
from people who use existing \proglang{R} functions and packages that
are not particularly scalable.  Examples include using the \func{optim}
function in base \proglang{R} to find the posterior mode, or the
\func{rmvnorm} function in the \pkg{mvtnorm} package for sampling from
a multivariate normal (MVN) random variable.  These are functions with which
most \proglang{R} users are familiar, so it is understandable that the
barrier to adoption of algorithms that fail when using them would be high.

Fortunately, there are alternatives to that are better
suited for running the BD algorithm on hierarchical models. These packages are listed in Table \ref{tab:packages}, and are
required to run the code in this vignette.  The \pkg{Matrix} and \pkg{sparseHessianFD} define
classes for working with sparse matrices, and the
\pkg{sparseMVN}, and \pkg{trustOptim} packages respectively provide sampling and
optimization routines that are designed to work with sparse matrices.
Those packages are not strictly necessary to use BD, but this package
(\pkg{bayesGDS}) implements only the rejection sampling phase.  The
\pkg{trustOptim}, \pkg{sparseHessianFD} and \pkg{sparseMVN} packages were
initially written as complements to \pkg{bayesGDS}, with the BD
algorithm in mind.


\begin{table}[htb]
  \centering
  \begin{tabular}{p{1in} p{3.3in} p{1.1in}}
    Package&Specific use for the BD algorithm&Relevant lines in
                                               BD Algorithm\\
\hline
    \pkg{Matrix}&Defines classes and methods for sparse matrices&3, 11, 32\\
    \hline
    \pkg{sparseHessianFD}&Numerical estimation of sparse Hessians when
                     the gradient and sparsity pattern are known&Setup, 3, 11, 32\\
    \hline
    \pkg{sparseMVN}&Sampling from, and to computing the
               log density of, a multivariate normal (MVN) random
               variable for which either the covariance or precision
               matrix is sparse. &11, 12, 32, 33\\
    \hline
    \pkg{trustOptim}&Nonlinear optimization with a gradient-based
                      stopping rule.  Efficient for objective functions with sparse Hessians&3\\
    \hline
    \pkg{bayesGDS}&Rejection sampling phase of the BD algorithm, and
                    computing marginal likelihood of the data under
                    the model&20-37\\
    \hline
    \pkg{doParallel}&Multiple core, shared memory parallelization for
                      sampling from proposal and posterior
                      distributions&11-12, 31-35\\
    \hline
    \pkg{plyr}&Parallel application of a function to a row or column of a matrix&12\\
    \hline
  \end{tabular}
  \caption{\proglang{R} packages used in this vignette for
    implementing the BD algorithm}
  \label{tab:packages}
\end{table}



\section{Some background}

The goal is to sample a
parameter vector $\theta$ from a posterior density $\pi(\theta|y)$,
where $\pi(\theta)$ is the prior on $\theta$, $f(y|\theta)$ is the data
likelihood conditional on $\theta$, and $\mathcal{L}(y)$ is the marginal
likelihood of the data.  Let $\mathcal{D}(\theta,y)$ be the joint
density of the data and the parameters. Therefore,
\begin{align}
\pi(\theta|y)=\frac{f(y|\theta)\pi(\theta)}{\mathcal{L}(y)}=\frac{\mathcal{D}(\theta,y)}{\mathcal{L}(y)}
\end{align}

The model is fully specified by $\mathcal{D}(\theta,y)$, which is
equivalent to the posterior density, up to a normalizing constant $1/\Ly$.

If the heterogeneous units are conditionally independent, the data likelihood can be factored as
\begin{align}
f(y|\theta)=\prod_{i=1}^Nf_i\left(y_i|\beta_i,\alpha\right)
\end{align}

where $i$ indexes households. Each $y_i$ is a vector of observed data, each $\beta_i$ is a vector of
heterogeneous parameters, and $\alpha$ is a vector of homogeneous population-level
parameters.  The $\beta_i$ are distributed across the population of
households according to a mixing distribution $\pi(\beta_i|\alpha)$,
which also serves as the prior on each $\beta_i$.  The elements of $\alpha$
may influence either the household-level data likelihoods, or the
mixing distribution (or both).  In this example, $\theta$ includes all
$\beta_1\mathellipsis\beta_N$ and all elements of $\alpha$.

The prior itself can be factored as
\begin{align}
\pi(\theta)=\prod_{i=1}^N\pi_i(\beta_i|\alpha)\times\pi(\alpha).
\end{align}
Thus, the log posterior density is written as
\begin{align}
\log \pi(\beta,\alpha|y)&=\sum_{i=1}^N\left[\log f(y_i|\beta_i)+\log\pi(\beta_i|\alpha)\right]+\log\pi(\alpha)-\log\mathcal{L}(y)
\end{align}

$\log\Dy$ is equivalent to $\log \pi(\beta,\alpha|y)$, without the
$\log\Ly$ term.

\subsection{Sparse Hessians}\label{sec:sparseHessians}

An implication of the conditional independence
assumption is that the cross-partial derivatives of the unnormalized
log posterior density,
$\dfrac{\partial^2\log\Dy}{\beta_i\beta_j}$ are zero for all $i\neq
j$. As the number of households in the dataset increases, the number
of elements in the Hessian matrix increases quadratically, but the
number of \emph{non-zero} elements increases only linearly.  Thus, the
Hessian becomes sparser as the data set gets larger.

A sparse matrix is one that has a relatively small number of nonzero
elements.  A sparse matrix can be represented by only the nonzero
values, and the row and column indices of those values.  All of the other
elements are known to be zero, so they do not need to be stored
explicitly.  Thus, the amount of memory required to store a sparse
matrix grows with the number of nonzero elements, as opposed to the
product of the number of rows and columns.  Also, linear algebra
operations are more efficient on compressed sparse matrices, because
operation on the nonzero elements can be ignored.  These computational
advantages come into play in nearly all of the steps of the BD
algorithm.  Although the sparsity of the Hessian of the log posterior
density is not a requirement for the BD algorithm, it is that sparsity
that makes the algorithm scalable \citep[Sec. 4]{BraunDamien2015}.



As an example, suppose we have a hierarchical model with $N$ heterogeneous units,
each with a parameter vector of length $k$.  Also, assume that there
are $p$ population-level parameters or hyperparameters.

The \emph{sparsity pattern} of the Hessian depends on how the variables are
ordered within the vector. One such ordering is to group all of the
coefficients for each unit together.
\begin{align}
\beta_{11},...,\beta_{1k},\beta_{21},...,\beta_{2k},...~,~...~,~\beta_{N1}~,~...~,~\beta_{Nk},\mu_1,...,\mu_p
\end{align}

In this case, the Hessian has a "block-arrow" structure.  For example,
if $N=6$, $k=2$, and $p=2$, then there are
14 total variables, and the Hessian will have the sparsity
pattern in Figure \ref{fig:pattern1}.

Another option would be to group the coefficients by covariate.
\begin{align}
\beta_{11},...,\beta_{1N},\beta_{21},...,\beta_{2N},...,...,\beta_{k1},...,\beta_{kN},\mu_1,...,\mu_p
\end{align}

Now the Hessian has an "off-diagonal" sparsity pattern, as in Figure \ref{fig:pattern2}.

\begin{figure}
  \begin{subfigure}[t]{.5\linewidth}
\begin{knitrout}
\definecolor{shadecolor}{rgb}{0.969, 0.969, 0.969}\color{fgcolor}\begin{kframe}
\begin{verbatim}
# 14 x 14 sparse Matrix of class "lgCMatrix"
#                                  
#  [1,] | | . . . . . . . . . . | |
#  [2,] | | . . . . . . . . . . | |
#  [3,] . . | | . . . . . . . . | |
#  [4,] . . | | . . . . . . . . | |
#  [5,] . . . . | | . . . . . . | |
#  [6,] . . . . | | . . . . . . | |
#  [7,] . . . . . . | | . . . . | |
#  [8,] . . . . . . | | . . . . | |
#  [9,] . . . . . . . . | | . . | |
# [10,] . . . . . . . . | | . . | |
# [11,] . . . . . . . . . . | | | |
# [12,] . . . . . . . . . . | | | |
# [13,] | | | | | | | | | | | | | |
# [14,] | | | | | | | | | | | | | |
\end{verbatim}
\end{kframe}
\end{knitrout}
\caption{A ``block-arrow'' sparsity pattern}\label{fig:pattern1}
\end{subfigure}
\begin{subfigure}[t]{.5\linewidth}
\begin{knitrout}
\definecolor{shadecolor}{rgb}{0.969, 0.969, 0.969}\color{fgcolor}\begin{kframe}
\begin{verbatim}
# 14 x 14 sparse Matrix of class "lgCMatrix"
#                                  
#  [1,] | . . . . . | . . . . . | |
#  [2,] . | . . . . . | . . . . | |
#  [3,] . . | . . . . . | . . . | |
#  [4,] . . . | . . . . . | . . | |
#  [5,] . . . . | . . . . . | . | |
#  [6,] . . . . . | . . . . . | | |
#  [7,] | . . . . . | . . . . . | |
#  [8,] . | . . . . . | . . . . | |
#  [9,] . . | . . . . . | . . . | |
# [10,] . . . | . . . . . | . . | |
# [11,] . . . . | . . . . . | . | |
# [12,] . . . . . | . . . . . | | |
# [13,] | | | | | | | | | | | | | |
# [14,] | | | | | | | | | | | | | |
\end{verbatim}
\end{kframe}
\end{knitrout}
\caption{An ``off-diagonal'' sparsity pattern.}\label{fig:pattern2}
\end{subfigure}
\caption{Examples of sparsity patterns for a hierarchical model.  The
  pattern depends on the ordering of the coefficients. The
  \class{lgCMatrix} class is defined in the \pkg{Matrix} package.}
\end{figure}


In both cases, the number of non-zeros is the same.  There are
196 elements in this symmetric matrix, but only  76 are
non-zero, and only 45 values are unique.  Although in this
example the reduction in
resource consumption from using a sparse matrix structure for the Hessian may be
modest, consider what would happen if $N=1000$ instead.  In that case,
there are $2,002$
variables in the problem, and more than $4$ million
elements in the Hessian.  However, only $12,004$ of those elements are
non-zero.  If we work with only the lower triangle of the Hessian we
only need to work with
only $7,003$ values.

As discussed in Section 4 of Braun and Damien (2015), the sparsity of
the Hessian is what makes the method scalable, in terms of the number
of heterogeneous units in the data set.  Each additional unit
adds $k$ rows and columns to the Hessian, so the number of formal
elements increases quadratically with $N$.  However, in terms of
\emph{non-zero} elements, each unit adds a $k\times k$ block on the
diagonal (correlation of variables within a unit), and a block in each
margin (correlation between unit-level and population-level
variables).  Thus, the number of non-zero elements grows linearly, not
quadratically, with $N$.  Consequently, the complexity of many of the steps of the
algorithm, such as multiplying matrices, generating Cholesky factors,
and solving sparse linear systems, grow linearly as well.


\subsection{Nonlinear optimization}\label{sec:optimization}

Finding the posterior mode $\theta^*$ can be a hard problem when there
is a high number of variables, especially when using standard
optimization routines, like the \func{optim} function in base
\proglang{R}, and others that are described in the CRAN Task View on
Optimization.  There are two common problems with these algorithms:

\paragraph{Premature termination}  Many algorithms apply a stopping rule that is based on
whether the optimizer is making "sufficient progress'' (note the
\funcarg{abstol} and \funcarg{reltol} control arguments for \func{optim}). If the
objective function does not improve by some minimum amount, the
algorithm believes that it has converged to a local optimum. This can happen before the gradient of the
objective function is sufficiently flat.  For an unconstrained objective
function, the \emph{only} appropriate stopping rule is whether a
function of the gradient (e.g., the Euclidean norm) is numerically
zero.  Any optimization routine that stops before that is stopping
prematurely, and the resulting normal approximations to the
posterior density would be invalid.

\paragraph{Poor scalability}   Search methods like Nelder-Mead (the default algorithm for
\func{optim}) are inefficient with a
massive number of parameters because the search space is large, and
they do not exploit information about slope and curvature to accelerate convergence.  Conjugate gradient (\method{CG}) and
quasi-Newton (e.g., \method{BFGS}) do use
gradient information, with \method{BFGS} tracing out the curvature of
the function by using successive gradients to approximate the inverse
Hessian.  However, because \method{BFGS} stores the entire dense
inverse Hessian, its use is resource-intensive when the number of
parameters is large.   \method{CG} and limited-memory \method{BFGS} methods do not store the full Hessian
(or its inverse), so they can be more suited for large-scale
problems. However, like \method{BFGS}, they are not certain to
approximate the curvature of the objective function accurately at any
particular iteration, especially if the function is not convex \citep{R_trustOptim}.

A better choice of nonlinear optimizer would be one that uses exact calculations of the gradient and Hessian,
remains scalable for large problems, stops only when the norm of the
gradient is numerically zero, and is stable when passing through
regions in which the surface of the objective function is flat.  The \func{trust.optim} function
in the \pkg{trustOptim} package meets those criteria.
\Citet{R_trustOptim} describes the many advantages of the
\func{trust.optim} function.  Two advantages of note are:

\begin{enumerate}
\item convergence only when the norm of the
  gradient is sufficiently close to zero\footnote{There is an
    exception.  If the trust region gets too small for any further
    progress to be made, \func{trust.optim} will stop.  This may
    happen if the norm of the gradient is larger than the specified
    tolerance, but still pretty small.  In that case, there is no
    problem.  If the trust region radius is nearly zero, but the
    gradient is nowhere close to zero, then there is some kind of
    problem that requires further investigation.}; and
\item the ability to use exact Hessian information that is stored as a
  compressed sparse matrix.
\end{enumerate}

The \func{trust.optim} function supports three optimization routines:
two quasi-Newton methods (\method{SR1} and \method{BFGS}; see
\citet{NocedalWright2006}), and a \method{Sparse} method.  All three
methods require the user to supply a function that returns the value
of the objective function, \emph{and} a function that returns the
gradient.  The \method{Sparse} method requires an additional function
that returns the Hessian as a \class{dgCMatrix} object.  The
\class{dgCMatrix} class is defined in the \pkg{Matrix} package, and is
one of several classes for the storage of, and operation on, sparse
matrices.

Be sure that the Hessian matrix function returns a
\class{dgCMatrix} matrix, and not a structured matrix like
\class{dsCMatrix} (symmetric) or \class{dtCMatrix} (triangular); a
row-oriented matrix like \class{dgRMatrix}; or even a dense matrix
like \class{dgeMatrix} or a base \proglang{R} \class{matrix}.
\func{trust.optim} accepts only \class{dgCMatrix}, even if the matrix
is numerically dense.


\subsection{Computing derivatives}\label{sec:derivatives}


Nearly all reasonable choices of a nonlinear optimization algorithm
require the user to provide a the gradient of the objective function.
The \method{Sparse} method for \func{trust.optim} requires the Hessian as well.  The Hessian is
also required to approximate the posterior density with a MVN
distribution around the posterior mode (see Section \ref{sec:MVN}).

For the purposes of the BD algorithm, there are two "good" ways to
compute a derivative. The first is to derive
it analytically, and write a function to compute it.  This approach is
straightforward, but it can be tedious and error-prone for complicated models.

The second is to use automatic,
or algorithmic, differentiation (AD).  In short, AD generates
code for the derivative by applying the chain rule on the same sequence of operations
that computes the objective function.  There are a number of different approaches to
implementing AD, and AD libraries are available for
many programming languages.  However, as of now, there are none for \proglang{R} that are
well-suited for a general class of Bayesian hierarchical models.  For
\proglang{R} users, we believe that coding the objective function in \proglang{C++} using the
\pkg{CppAD} library, and interfacing with \proglang{R} using \pkg{Rcpp}, is the best option
at the moment.  How to do this will be the subject of a future
vignette.  What matters is that functions that return the gradient and
Hessian of the log posterior density are available, and that they are sufficiently
accurate.  The advantage of both analytic and AD derivatives is that they are ``exact.''

We do not recommend estimating the gradient by numerical approximation via finite
differencing (FD).  FD involves computing
$\partial f/\partial x_j\approx\left[f\left(x_j+h\right)-f\left(x_j\right)\right]/h$,
or some variation thereof, for each of the $j=1...J$ variables,
using an arbitrarily small $h$ as a ``perturbation factor.''  As $h\rightarrow 0$, this estimated
difference approaches the gradient.  Not only are FD methods are highly vulnerable
to numerical precision problems, but the complexity of the method
grows with the number of variables   Thus, FD is not a reasonable
option for estimating the gradient
when the number of variables is large.  The time to compute the
gradient using AD, on the
other hand, is only a small fixed multiple of the time to compute the
objective function, regardless of the number of variables
\citep{GriewankWalther2008}.

The computational cost of computing a dense Hessian using FD is
quadratic in the number of variables, and the numerical precision
problems are even more pronounced than for a gradient.  Nevertheless,
one can use FD to estimate the Hessian if the Hessian is sparse, \emph{and} the
sparsity pattern is known in advance, \emph{and} the gradient is
exact (either derived analytically or computed using AD).  The
\pkg{sparseHessianFD} package defines an \proglang{R} class for doing
this (the algorithms are based on \citet{ColemanGarbow1985} and
\citet{ColemanGarbow1985b}). An object
of class \class{sparseHessianFD} is constructed from
functions that return the value of the log density and its gradient; any additional
arguments that are passed to these functions; and the row
and column indices of the non-zero elements of the lower triangle of
the Hessian.  The \class{sparseHessianFD} object contains member
functions not only for the log density and the gradient, but also for
the sparse Hessian, as a \class{dgCMatrix}
object.  See the \pkg{sparseHessianFD} documentation and vignette for
more details.

The \class{sparseHessianFD} class is useful for
hierarchical models because the
sparsity pattern is predictable.
Even if the Hessian were derived analytically, it still may be faster
to use \class{sparseHessianFD} for
repeated estimation, because of the way the package exploits sparsity.  Given the typical sparsity pattern of a hierarchical model, the time to estimate
a Hessian using \pkg{sparseHessianFD} does not grow with the number of
heterogeneous units, making it a scalable way of
estimating sparse Hessians for large datasets.

The trade-off from using the \pkg{sparseHessianFD} package is that the
Hessian is still a numerical approximation.  We cannot guarantee that
this approximation is "good enough" for all cases.  However, the
performance of an optimization algorithm is likely to be more sensitive to
numerical imprecision in the gradient than in the Hessian. Thus, we
are comfortable working with finite-differenced sparse Hessians, even though
we will not use finite-differenced gradients.  This approach will
almost certainly fail if the gradient itself is estimated using FD.
In that case, the estimate of the Hessian would be a finite difference of finite
differences, with too much numerical imprecision to be of much
value.


\subsection{High-dimensional MVN distribution}\label{sec:MVN}

Another potential source of poor scalability is in sampling from, and
computing the log density of, a multivariate normal (MVN)
distribution. The \func{rmvnorm} and \func{dmvnorm} functions in the
\pkg{mvtnorm} package \citep{R_mvtnorm} are inefficient
in the context of the BD algorithm for three reasons.

\begin{enumerate}
\item They require the covariance matrix as one of the arguments, meaning that
the negative Hessian must be inverted explicitly.
\item They do not
exploit the sparsity of the Hessian for computational gain; and
\item The \func{rmvnorm} and \func{dmvnorm} functions factor the
  covariance matrix every time the function is called, even if the matrix has not changed.
\end{enumerate}

The \pkg{sparseMVN} package addresses these issues.  The \func{rmvn.sparse} and
\func{dmvn.sparse} functions take, as one of the arguments, the Cholesky
decomposition of either a sparse covariance or sparse precision
matrix.  Thus, the user can use either the covariance
or precision matrix, depending on which is more convenient.  The
Cholesky decomposition must be a \class{Cholesky} object, computed
using the \func{Cholesky} function, both of which are defined in the
\pkg{Matrix} package.  The size
of the sparse Hessian grows only linearly with the size of the
dataset, the \pkg{sparseMVN} package is a scalable alternative for working
with an MVN.  More details are available in the \pkg{sparseMVN}
documentation and vignette, as well as in the example in \ref{sec:runAlgorithm}.

\section{Example:  hierarchical binary choice}\label{sec:example}

Before going into the details of how to put all of the pieces of the
BD algorithm together, let's
consider the following example of a log posterior density function
with a sparse Hessian.
 Suppose we have a dataset of $N$ households, each with $T$
 opportunities to purchase a particular product.  Let $y_i$ be the
 number of times household $i$ purchases the product, out of the $T$
 purchase opportunities.  Furthermore, let $p_i$ be the probability of
 purchase; $p_i$ is the same for all $T$ opportunities, so we can treat $y_i$ as a binomial random variable.  The purchase probability
 $p_i$ is heterogeneous, and depends on both $k$ continuous covariates
 $x_i$, and a heterogeneous coefficient vector $\beta_i$, such that
\begin{align}
  p_i=\frac{\exp(x_i'\beta_i)}{1+\exp(x_i'\beta_i)},~i=1 ... N
\end{align}

The coefficients can be thought of as sensitivities to the covariates,
and they are distributed across the population of households following
a multivariate normal distribution with mean $\mu$ and covariance
$\Sigma$.   We assume that we know $\Sigma$, but we do not know $\mu$.
Instead, we place a multivariate normal prior on $\mu$, with mean $0$
and covariance $\Omega_0$.  Thus, each $\beta_i$, and $\mu$ are
$k-$dimensional vectors, and the total number of unknown variables in
the model is $(N+1)k$.

In this model, we will make an assumption of \emph{conditional
independence} across households.   A household's purchase count $y_i$
depends on that household's $\beta_i$, but not the
parameters of any other household, $\beta_j$, conditional on other
population level parameters.  Since $\mu$ and $\Sigma$ depend on
$\beta_i$ for \emph{all} households, we cannot say that $y_i$ and $y_j$ are
truly independent.  A change in $\beta_i$ affects $\mu$ and
$\Sigma$, which in turn affect $\beta_j$ for some other household
$j$.  However, if we condition on $\mu$ and $\Sigma$, then $y_i$ and $y_j$
are independent, so we describe the data likelihoods as conditionally independent.

This conditional independence assumption is what allows us to write
the joint likelihood of the data as a product of individual-level
probability models.  Therefore, the log posterior density, ignoring any normalization constants, is
\begin{align}
  \log \pi(\beta_{1:N},\mu|Y, X, \Sigma_0,\Omega_0)=\sum_{i=1}^Np_i^{y_i}(1-p_i)^{T-y_i}
  -\frac{1}{2}\left(\beta_i-\mu\right)'\Sigma^{-1}\left(\beta_i-\mu\right)
-\frac{1}{2}\mu'\Omega_0^{-1}\mu
\end{align}

\subsection{Functions, sample data, and priors for the example}

The \pkg{bayesGDS} package includes a simulated dataset for $N$ households and $k$
covariates per household.  There are $q$ population-level parameters,
which for this example happens to equal $k$.  A household makes $Y$ purchases out of $T$
opportunities.  The purchase probability for each household depends on
a covariate matrix $X$.  So let's start by loading the data, and setting hyperprior values for $\Sigma^{-1}$ and $\Omega^{-1}$.

\begin{knitrout}
\definecolor{shadecolor}{rgb}{0.969, 0.969, 0.969}\color{fgcolor}\begin{kframe}
\begin{alltt}
\hlkwd{data}\hlstd{(binary_small)}
\hlstd{binary} \hlkwb{<-} \hlstd{binary_small} \hlcom{#rename for brevity}
\hlkwd{str}\hlstd{(binary)}
\end{alltt}
\begin{verbatim}
# List of 3
#  $ Y: int [1:20] 101 69 108 86 92 92 96 149 58 84 ...
#  $ X: num [1:2, 1:20] -0.07926 -0.03255 0.22043 0.00997 0.01828 ...
#  $ T: num 200
\end{verbatim}
\begin{alltt}
\hlstd{N} \hlkwb{<-} \hlkwd{length}\hlstd{(binary[[}\hlstr{"Y"}\hlstd{]])}
\hlstd{k} \hlkwb{<-} \hlkwd{NROW}\hlstd{(binary[[}\hlstr{"X"}\hlstd{]])}
\hlstd{q} \hlkwb{<-} \hlstd{k}
\hlstd{nvars} \hlkwb{<-} \hlkwd{as.integer}\hlstd{(N}\hlopt{*}\hlstd{k} \hlopt{+} \hlstd{q)}
\hlstd{priors} \hlkwb{<-} \hlkwd{list}\hlstd{(}\hlkwc{inv.Sigma} \hlstd{=} \hlkwd{diag}\hlstd{(k),} \hlcom{##rWishart(1,k+5,diag(k))[,,1],}
               \hlkwc{inv.Omega} \hlstd{=} \hlkwd{diag}\hlstd{(k))}
\hlkwd{data.frame}\hlstd{(}\hlkwc{parameter} \hlstd{=} \hlkwd{c}\hlstd{(}\hlstr{"N"}\hlstd{,}\hlstr{"k"}\hlstd{,}\hlstr{"q"}\hlstd{),}
           \hlkwc{value} \hlstd{=}  \hlkwd{c}\hlstd{(N, k, q))}
\end{alltt}
\begin{verbatim}
#   parameter value
# 1         N    20
# 2         k     2
# 3         q     2
\end{verbatim}
\end{kframe}
\end{knitrout}


The function \func{binary.f} returns the unnormalized log
posterior density of this model, $\log\Dy$, evaluated at the vector passed as the
first argument.  This function takes two additional named arguments,
\funcarg{data} and \funcarg{priors}.  The
function \funcarg{binary.grad} returns the gradient, and \funcarg{binary.hess} returns a
Hessian, in a sparse compressed format.


\begin{knitrout}
\definecolor{shadecolor}{rgb}{0.969, 0.969, 0.969}\color{fgcolor}\begin{kframe}
\begin{alltt}
\hlstd{start} \hlkwb{<-} \hlkwd{rnorm}\hlstd{(nvars)} \hlcom{## random starting values}
\hlstd{f} \hlkwb{<-} \hlkwd{binary.f}\hlstd{(start,} \hlkwc{data}\hlstd{=binary,} \hlkwc{priors}\hlstd{=priors)}
\hlstd{f}
\end{alltt}
\begin{verbatim}
# [1] -2799.4
\end{verbatim}
\begin{alltt}
\hlstd{df} \hlkwb{<-} \hlkwd{binary.grad}\hlstd{(start,} \hlkwc{data}\hlstd{=binary,} \hlkwc{priors}\hlstd{=priors)}
\hlkwd{str}\hlstd{(df)}
\end{alltt}
\begin{verbatim}
#  num [1:42] -1.066 0.684 -2.505 0.336 0.442 ...
\end{verbatim}
\begin{alltt}
\hlstd{d2f} \hlkwb{<-} \hlkwd{binary.hess}\hlstd{(start,} \hlkwc{data}\hlstd{=binary,} \hlkwc{priors}\hlstd{=priors)}
\hlkwd{print}\hlstd{(d2f[}\hlnum{1}\hlopt{:}\hlnum{6}\hlstd{,}\hlnum{1}\hlopt{:}\hlnum{6}\hlstd{],} \hlkwc{digits}\hlstd{=}\hlnum{3}\hlstd{)}
\end{alltt}
\begin{verbatim}
# 6 x 6 sparse Matrix of class "dgCMatrix"
#                                                     
# [1,]  -1.314  -0.129 .       .       .       .      
# [2,]  -0.129  -1.053 .       .       .       .      
# [3,] .       .        -3.392  -0.108 .       .      
# [4,] .       .        -0.108  -1.005 .       .      
# [5,] .       .       .       .        -1.017  -0.221
# [6,] .       .       .       .        -0.221  -3.937
\end{verbatim}
\end{kframe}
\end{knitrout}

\subsection{Using sparseHessianFD when the Hessian is unknown}

Hessians can be difficult to derive analytically, but Hessians for hierarchical models
have predictable sparsity patterns. The
\pkg{sparseHessianFD} package simplifies numerical approximation of a
sparse Hessian, and should be used if exact
methods are unavailable (see Section \ref{sec:derivatives}.   The
\func{sparseHessianFD.new} function is a wrapper around the
constructor for the \class{sparseHessianFD} class, and returns an
object of that class.  But before calling \func{sparseHessianFD.new},
we need to get the sparsity pattern of the lower triangle of the
Hessian of $\log\Dy$. The sparsity pattern is defined
by integer vectors of the row and column indices of the nonzero elements.

A straightforward
way to generate the sparsity pattern is to build a sparse integer or logical matrix,
and then use the \func{Matrix.to.Coord} function from
\pkg{sparseHessianFD}.  The \func{Matrix.to.Coord} function returns a
list of two integer vectors that can be passed to \func{sparseHessianFD.new}.

\begin{knitrout}
\definecolor{shadecolor}{rgb}{0.969, 0.969, 0.969}\color{fgcolor}\begin{kframe}
\begin{alltt}
\hlkwd{require}\hlstd{(sparseHessianFD,} \hlkwc{quietly}\hlstd{=}\hlnum{TRUE}\hlstd{)}
\hlstd{hs} \hlkwb{<-} \hlkwd{Matrix}\hlstd{(}\hlnum{0}\hlstd{, nvars, nvars)}
\hlkwa{for} \hlstd{(i} \hlkwa{in} \hlnum{1}\hlopt{:}\hlstd{(N} \hlopt{+} \hlnum{1}\hlstd{)) \{}
    \hlcom{## range of row / col indices of block diagonal}
    \hlstd{rng} \hlkwb{<-} \hlstd{((i}\hlopt{-}\hlnum{1}\hlstd{)}\hlopt{*}\hlstd{k}\hlopt{+}\hlnum{1}\hlstd{)}\hlopt{:}\hlstd{(k}\hlopt{*}\hlstd{i)}
    \hlstd{hs[rng, rng]} \hlkwb{<-} \hlkwd{tril}\hlstd{(}\hlkwd{Matrix}\hlstd{(}\hlnum{1}\hlstd{,k,k))} \hlcom{## lower triangle}
\hlstd{\}}
\hlstd{hs[N}\hlopt{*}\hlstd{k} \hlopt{+} \hlnum{1}\hlopt{:}\hlstd{q,} \hlnum{1}\hlopt{:}\hlstd{(N}\hlopt{*}\hlstd{k)]} \hlkwb{<-} \hlnum{1} \hlcom{## bottom margin}
\hlstd{hsNZ} \hlkwb{<-} \hlkwd{Matrix.to.Coord}\hlstd{(hs)}
\hlkwd{str}\hlstd{(hsNZ)}
\end{alltt}
\begin{verbatim}
# List of 2
#  $ rows: int [1:143] 1 2 41 42 2 41 42 3 4 41 ...
#  $ cols: int [1:143] 1 1 1 1 2 2 2 3 3 3 ...
\end{verbatim}
\end{kframe}
\end{knitrout}

Now we can construct the \class{sparseHessianFD} object.

\begin{knitrout}
\definecolor{shadecolor}{rgb}{0.969, 0.969, 0.969}\color{fgcolor}\begin{kframe}
\begin{alltt}
\hlstd{FD} \hlkwb{<-} \hlkwd{sparseHessianFD.new}\hlstd{(start, binary.f, binary.grad,}
                          \hlkwc{rows}\hlstd{=hsNZ[[}\hlstr{"rows"}\hlstd{]],} \hlkwc{cols}\hlstd{=hsNZ[[}\hlstr{"cols"}\hlstd{]],}
                          \hlkwc{data}\hlstd{=binary,} \hlkwc{priors}\hlstd{=priors)}
\end{alltt}
\end{kframe}
\end{knitrout}

One advantage to using \class{sparseHessianFD} is that any
additional arguments that need to be passed to the objective function are stored within the object. This means that we do not need to repeatedly pass
the data and prior arguments when computing the log posterior density
and its derivatives.

\begin{knitrout}
\definecolor{shadecolor}{rgb}{0.969, 0.969, 0.969}\color{fgcolor}\begin{kframe}
\begin{alltt}
\hlstd{f} \hlkwb{<-} \hlstd{FD}\hlopt{$}\hlkwd{fn}\hlstd{(start)}
\hlstd{df} \hlkwb{<-} \hlstd{FD}\hlopt{$}\hlkwd{gr}\hlstd{(start)}
\hlstd{hess} \hlkwb{<-} \hlstd{FD}\hlopt{$}\hlkwd{hessian}\hlstd{(start)}
\end{alltt}
\end{kframe}
\end{knitrout}

As a check, we see that the upper-left
  corner of the Hessian is sparse (the entire Hessian is too big to
  print), and \code{FD\$hessian} returns the
  same matrix as \code{binary.hess}.

\begin{knitrout}
\definecolor{shadecolor}{rgb}{0.969, 0.969, 0.969}\color{fgcolor}\begin{kframe}
\begin{alltt}
\hlkwd{print}\hlstd{(hess[}\hlnum{1}\hlopt{:}\hlnum{6}\hlstd{,}\hlnum{1}\hlopt{:}\hlnum{6}\hlstd{],} \hlkwc{digits}\hlstd{=}\hlnum{3}\hlstd{)}
\end{alltt}
\begin{verbatim}
# 6 x 6 sparse Matrix of class "dgCMatrix"
#                                                     
# [1,]  -1.314  -0.129 .       .       .       .      
# [2,]  -0.129  -1.053 .       .       .       .      
# [3,] .       .        -3.392  -0.108 .       .      
# [4,] .       .        -0.108  -1.005 .       .      
# [5,] .       .       .       .        -1.017  -0.221
# [6,] .       .       .       .        -0.221  -3.937
\end{verbatim}
\begin{alltt}
\hlkwd{all.equal}\hlstd{(hess, d2f,} \hlkwc{tolerance} \hlstd{=} \hlnum{1e-7}\hlstd{)}
\end{alltt}
\begin{verbatim}
# [1] TRUE
\end{verbatim}
\end{kframe}
\end{knitrout}

\section{Running the algorithm}\label{sec:runAlgorithm}

Now it's time to estimate the posterior distributions of the model parameters.

\subsection{Finding the posterior mode}

For reasons discussed in Section \ref{sec:optimization}, we use the
\pkg{trustOptim} package to find the optimum of $\log\Dy$.  For the \funcarg{fn}, \funcarg{gr} and \funcarg{hs}
arguments, we provide the member functions of the FD object.  Details
on the control list for \func{trust.optim} are available in the \pkg{trustOptim} documentation.
The most important option to mention here is that
\funcarg{function.scale.factor} must be positive for minimization, and
negative for maximization.

\begin{knitrout}
\definecolor{shadecolor}{rgb}{0.969, 0.969, 0.969}\color{fgcolor}\begin{kframe}
\begin{alltt}
\hlkwd{require}\hlstd{(trustOptim,} \hlkwc{quietly}\hlstd{=}\hlnum{TRUE}\hlstd{)}
\hlstd{opt} \hlkwb{<-} \hlkwd{trust.optim}\hlstd{(start,} \hlkwc{fn}\hlstd{=FD}\hlopt{$}\hlstd{fn,} \hlkwc{gr} \hlstd{= FD}\hlopt{$}\hlstd{gr,} \hlkwc{hs} \hlstd{= FD}\hlopt{$}\hlstd{hessian,}
                   \hlkwc{method} \hlstd{=} \hlstr{"Sparse"}\hlstd{,}
                   \hlkwc{control} \hlstd{=} \hlkwd{list}\hlstd{(}
                       \hlkwc{start.trust.radius}\hlstd{=}\hlnum{5}\hlstd{,} \hlkwc{stop.trust.radius} \hlstd{=} \hlnum{1e-7}\hlstd{,}
                       \hlkwc{prec}\hlstd{=}\hlnum{1e-7}\hlstd{,} \hlkwc{report.precision}\hlstd{=}\hlnum{1}\hlstd{,}
                       \hlkwc{maxit}\hlstd{=}\hlnum{500}\hlstd{,} \hlkwc{preconditioner}\hlstd{=}\hlnum{1}\hlstd{,}
                       \hlkwc{function.scale.factor}\hlstd{=}\hlopt{-}\hlnum{1}
                       \hlstd{)}
                   \hlstd{)}
\end{alltt}
\begin{verbatim}
# Beginning optimization
# 
# iter      f   nrm_gr                     status
#   1   2720.7   13.1     Continuing - TR expand
#   2   2693.4    0.2                 Continuing
#   3   2693.4    0.0                 Continuing
#   4   2693.4    0.0                 Continuing
# 
# Iteration has terminated
#   4   2693.4    0.0                    Success
\end{verbatim}
\begin{alltt}
\hlstd{theta.star} \hlkwb{<-} \hlstd{opt[[}\hlstr{"solution"}\hlstd{]]}
\hlstd{hess} \hlkwb{<-} \hlstd{opt[[}\hlstr{"hessian"}\hlstd{]]}
\end{alltt}
\end{kframe}
\end{knitrout}


\subsection{Defining proposal functions}

Next, we define functions to sample from a proposal density, and to
compute the log density of a proposal draw.  As discussed in Section
\ref{sec:MVN}, \func{rmvn.sparse} and \func{dmvn.sparse} require a mean
vector, and a Cholesky decomposition of either the covariance or
precision matrix, as separate arguments.  The \func{sample.GDS}
function in \pkg{bayesGDS} requires those parameters to be in a single
list.  That means we need some wrapper functions.

\begin{knitrout}
\definecolor{shadecolor}{rgb}{0.969, 0.969, 0.969}\color{fgcolor}\begin{kframe}
\begin{alltt}
\hlkwd{require}\hlstd{(sparseMVN,} \hlkwc{quietly}\hlstd{=}\hlnum{TRUE}\hlstd{)}
\hlstd{rmvn.sparse.wrap} \hlkwb{<-} \hlkwa{function}\hlstd{(}\hlkwc{n.draws}\hlstd{,} \hlkwc{params}\hlstd{) \{}
    \hlkwd{rmvn.sparse}\hlstd{(n.draws, params[[}\hlstr{"mean"}\hlstd{]], params[[}\hlstr{"CH"}\hlstd{]],} \hlkwc{prec}\hlstd{=}\hlnum{TRUE}\hlstd{)}
\hlstd{\}}
\hlstd{dmvn.sparse.wrap} \hlkwb{<-} \hlkwa{function}\hlstd{(}\hlkwc{d}\hlstd{,} \hlkwc{params}\hlstd{) \{}
    \hlkwd{dmvn.sparse}\hlstd{(d, params[[}\hlstr{"mean"}\hlstd{]], params[[}\hlstr{"CH"}\hlstd{]],} \hlkwc{prec}\hlstd{=}\hlnum{TRUE}\hlstd{)}
\hlstd{\}}
\end{alltt}
\end{kframe}
\end{knitrout}

The proposal mean is
the posterior mean.  The proposal precision is the negative Hessian,
times a scaling factor.  The \func{Cholesky} function is defined in
the \pkg{Matrix} package.  Do not use the base \proglang{R} function
\func{chol}, which is not designed for sparse matrices, and cannot be
used by the \pkg{sparseMVN} functions.


\begin{knitrout}
\definecolor{shadecolor}{rgb}{0.969, 0.969, 0.969}\color{fgcolor}\begin{kframe}
\begin{alltt}
\hlstd{scale} \hlkwb{<-} \hlnum{.96}
\hlstd{chol.hess} \hlkwb{<-} \hlkwd{Cholesky}\hlstd{(}\hlopt{-}\hlstd{scale}\hlopt{*}\hlstd{hess)}
\hlstd{prop.params} \hlkwb{<-} \hlkwd{list}\hlstd{(}\hlkwc{mean} \hlstd{= theta.star,} \hlkwc{CH} \hlstd{= chol.hess)}
\end{alltt}
\end{kframe}
\end{knitrout}



\subsection{Side note:  running the algorithm in parallel}

An advantage of BD over MCMC is that both proposal and posterior
samples can be generated in parallel.  There are a number of different
mechanisms available in \proglang{R} for running jobs in parallel.  One that we
think is easy to use is based on the \pkg{doParallel} package.  In the
next code chunk, I allocate 10 cores for parallel processing, and set
a random seed for simulating random variables (more on that later).
If you do not want to run the code in parallel, change the flag to
\code{run.par <- FALSE}.  This will change the number of allocated
cores to 1.

\begin{knitrout}
\definecolor{shadecolor}{rgb}{0.969, 0.969, 0.969}\color{fgcolor}\begin{kframe}
\begin{alltt}
\hlkwd{library}\hlstd{(doParallel,} \hlkwc{quietly}\hlstd{=}\hlnum{TRUE}\hlstd{)}
\hlstd{run.par} \hlkwb{<-} \hlnum{TRUE}
\hlkwa{if}\hlstd{(run.par)} \hlkwd{registerDoParallel}\hlstd{(}\hlkwc{cores}\hlstd{=}\hlnum{2}\hlstd{)} \hlkwa{else} \hlkwd{registerDoParallel}\hlstd{(}\hlkwc{cores}\hlstd{=}\hlnum{1}\hlstd{)}
\hlstd{seed.id} \hlkwb{<-} \hlnum{123}
\hlkwd{set.seed}\hlstd{(seed.id)}
\end{alltt}
\end{kframe}
\end{knitrout}




\subsection{Estimating the density of threshold values}
\Citet{BraunDamien2015} define the following:
\begin{align}
  \label{eq:logPhi}
  \Phitheta&=\displaystyle\frac{\Dy\cdot c_2}{\Gtheta \cdot c_1}\\
  c_1&=\mathcal{D}(y,\theta^*)\\
  c_2&=g(\theta^*)
\end{align}
where $\Gtheta$ is chosen such that $0 < \Phitheta \leq 1$.  The next step in the BD algorithm is to simulate an empirical
approximation to the cumulative distribution of $-\log\Phi(\theta|y)$.  We sample $M$
times from the proposal distribution, and compute
$\log\Phi(\theta|y)$ for each proposal sample.  The next code chunk
also uses the \func{aaply} function from the \pkg{plyr} package to compute
the log posterior density for each proposal draw.  This step will run in parallel if
\code{run.par == TRUE}.

\begin{knitrout}
\definecolor{shadecolor}{rgb}{0.969, 0.969, 0.969}\color{fgcolor}\begin{kframe}
\begin{alltt}
\hlstd{M} \hlkwb{<-} \hlnum{10000}  \hlcom{## proposal draws}
\hlstd{log.c1} \hlkwb{<-} \hlstd{FD}\hlopt{$}\hlkwd{fn}\hlstd{(theta.star)}
\hlstd{log.c2} \hlkwb{<-} \hlkwd{dmvn.sparse.wrap}\hlstd{(theta.star, prop.params)}
\hlstd{draws.m} \hlkwb{<-} \hlkwd{as}\hlstd{(}\hlkwd{rmvn.sparse.wrap}\hlstd{(M,prop.params),}\hlstr{"matrix"}\hlstd{)}
\hlstd{log.post.m} \hlkwb{<-} \hlstd{plyr::}\hlkwd{aaply}\hlstd{(draws.m,} \hlnum{1}\hlstd{, FD}\hlopt{$}\hlstd{fn,} \hlkwc{.parallel}\hlstd{=run.par)}
\hlstd{log.prop.m} \hlkwb{<-} \hlkwd{dmvn.sparse.wrap}\hlstd{(draws.m,} \hlkwc{params}\hlstd{=prop.params)}
\hlstd{log.phi} \hlkwb{<-} \hlstd{log.post.m} \hlopt{-} \hlstd{log.prop.m} \hlopt{+} \hlstd{log.c2} \hlopt{-} \hlstd{log.c1}
\hlstd{valid.scale} \hlkwb{<-} \hlkwd{all}\hlstd{(log.phi} \hlopt{<=} \hlnum{0}\hlstd{)}
\hlkwd{stopifnot}\hlstd{(valid.scale)}
\end{alltt}
\end{kframe}
\end{knitrout}

The last two lines in the previous code chunk are checks that
$\Phi(\theta|y)\leq 1$ for all of the proposal draws.  If the check fails, then
the proposal distribution is invalid.  If this happens, the proposal
distribution $\Gtheta$ may need to be more diffuse by reducing the
scaling factor on the negative Hessian.

If $M$ is too low, there may not be enough proposal draws to confirm
that the proposal distribution is sufficiently diffuse.  This can
cause problems later.  Also, making the proposal distribution
too "tight" may make it too hard to accept posterior samples.  We think that time invested in ``optimizing'' the
proposal distribution is not a well-used resource.


\subsection{Posterior draws via rejection sampling}

 \emph{Finally} we can start sampling from the posterior density,
 using functions in the \pkg{bayesGDS} package.  We should proceed to
 this step only if \code{valid.scale==TRUE}.

 The \func{sample.GDS} function is the ``workhorse'' function of
 \pkg{bayesGDS}.  Each call to \func{sample.GDS} collects posterior
 samples serially.  The argument \funcarg{n.draws} is the number of draws for that
 particular call to \func{sample.GDS}. The other required arguments to
 \func{sample.GDS} are the $M$-length vector of $\log\Phi(\theta|y)$,
 the posterior mode $\theta^*$; the function that returns the log
 posterior density; the functions that sample from, and compute the
 log density of, the proposal distribution; parameters of the proposal
 distribution.  See the \pkg{bayesGDS} package documentation for
 descriptions of the function arguments.

 The following code chunk runs \func{sample.GDS} on a simple processor.

\begin{knitrout}
\definecolor{shadecolor}{rgb}{0.969, 0.969, 0.969}\color{fgcolor}\begin{kframe}
\begin{alltt}
\hlstd{n.draws} \hlkwb{<-} \hlnum{5}  \hlcom{## total number of draws needed}
\hlstd{max.tries} \hlkwb{<-} \hlnum{100000}  \hlcom{## to keep sample.GDS from running forever}
\hlkwa{if} \hlstd{(}\hlopt{!}\hlstd{run.par) \{}
    \hlstd{draws} \hlkwb{<-} \hlkwd{sample.GDS}\hlstd{(}\hlkwc{n.draws} \hlstd{= n.draws,}
                        \hlkwc{log.phi} \hlstd{= log.phi,}
                        \hlkwc{theta.star} \hlstd{= theta.star,}
                        \hlkwc{fn.dens.post} \hlstd{= FD}\hlopt{$}\hlstd{fn,}
                        \hlkwc{fn.dens.prop} \hlstd{= dmvn.sparse.wrap,}
                        \hlkwc{fn.draw.prop} \hlstd{= rmvn.sparse.wrap,}
                        \hlkwc{prop.params} \hlstd{= prop.params,}
                        \hlkwc{report.freq} \hlstd{=} \hlnum{1}\hlstd{,} \hlkwc{announce}\hlstd{=}\hlnum{TRUE}\hlstd{)}
\hlstd{\}}
\end{alltt}
\end{kframe}
\end{knitrout}


The \func{foreach}
function runs \func{sample.GDS} in parallel on multiple processors.
Each instance of \func{sample.GDS} is responsible for a batch of
posterior draws.  If we need \variable{n.draws} samples from the
posterior, and want to run \variable{n.batch} batches in parallel
(say, one batch for each core), then each instance will collect
\variable{batch.size} samples (rounding up to the nearest integer). The return object of \func{foreach} is
a list, with each element being a return object of \func{sample.GDS}.
The order in which the \func{sample.GDS} objects are returned does not matter.
Samples can be identified by batch with the \funcarg{thread.id}
argument.  It is important to start each instance with a different
random seed, which we specify in the \funcarg{seed} argument.

When all processing cores have finished collecting their samples, we
combine the list elements using a Map-Reduce construct.

\begin{kframe}
\begin{alltt}
\hlkwa{if} \hlstd{(run.par) \{}
    \hlstd{n.batch} \hlkwb{<-} \hlnum{10}
    \hlstd{batch.size} \hlkwb{<-} \hlkwd{ceiling}\hlstd{(n.draws} \hlopt{/} \hlstd{n.batch)}
    \hlstd{draws.list} \hlkwb{<-} \hlkwd{foreach}\hlstd{(}\hlkwc{i}\hlstd{=}\hlnum{1}\hlopt{:}\hlstd{n.batch,} \hlkwc{.inorder}\hlstd{=}\hlnum{FALSE}\hlstd{)} \hlopt \hlkwd{sample.GDS}\hlstd{(}
        \hlkwc{n.draws} \hlstd{= n.draws,}
        \hlkwc{log.phi} \hlstd{= log.phi,}
        \hlkwc{post.mode} \hlstd{= theta.star,}
        \hlkwc{fn.dens.post} \hlstd{= FD}\hlopt{$}\hlstd{fn,}
        \hlkwc{fn.dens.prop} \hlstd{= dmvn.sparse.wrap,}
        \hlkwc{fn.draw.prop} \hlstd{= rmvn.sparse.wrap,}
        \hlkwc{prop.params} \hlstd{= prop.params,}
        \hlkwc{report.freq} \hlstd{=} \hlnum{1}\hlstd{,}
        \hlkwc{thread.id} \hlstd{= i,}
        \hlkwc{announce}\hlstd{=}\hlnum{TRUE}\hlstd{,}
        \hlkwc{seed}\hlstd{=}\hlkwd{as.integer}\hlstd{(seed.id}\hlopt{*}\hlstd{i))}
    \hlcom{## combine results from each batch}
    \hlstd{draws} \hlkwb{<-} \hlkwd{Reduce}\hlstd{(}\hlkwa{function}\hlstd{(}\hlkwc{x}\hlstd{,}\hlkwc{y}\hlstd{)} \hlkwd{Map}\hlstd{(rbind,x,y), draws.list)}
\hlstd{\}}
\end{alltt}
\end{kframe}

There are almost certainly ways to run \func{sample.GDS} on
distributed memory platforms (e.g., MPI, Amazon, etc), but we have not
tried them yet.  If you are able to do that successfully, please share
your experiences.

One problem with this batch-oriented parallel sampling scheme is that
the algorithm does not end until all of the samples are collected from
all of the batches.  This means that some batches will finish before
others.  Suppose we are running \func{sample.GDS} on two processing
cores.  It may happen that one core finishes, but the other core still
has more than one sample to go.  There is no way for the first core to
help the second one.  It's probably possible, but it is not yet
implemented in this package.


\subsection{The output}

Let's take a look at the \func{sample.GDS} output.

\begin{knitrout}
\definecolor{shadecolor}{rgb}{0.969, 0.969, 0.969}\color{fgcolor}\begin{kframe}
\begin{alltt}
\hlkwd{str}\hlstd{(draws)}
\end{alltt}
\begin{verbatim}
# List of 7
#  $ draws         : num [1:50, 1:42] -0.182 -1.32 -1.636 -1.297 -1.663 ...
#  $ counts        : num [1:10, 1:5] 1 1 1 1 1 1 1 1 1 1 ...
#  $ gt.1          : int [1:10, 1:5] 0 0 0 0 0 0 0 0 0 0 ...
#  $ log.post.dens : num [1:10, 1:5] -2713 -2716 -2714 -2717 -2709 ...
#  $ log.prop.dens : num [1:10, 1:5] -48.7 -51.1 -49.1 -51.9 -44.5 ...
#  $ log.thresholds: num [1:10, 1:5] 2.06 1.41 2.48 1.39 1.61 ...
#  $ log.phi       : num [1:10, 1:5] -0.803 -1.01 -0.654 -0.958 -0.608 ...
\end{verbatim}
\end{kframe}
\end{knitrout}

The \funcarg{draws} element has a posterior sample in each row.  Each
column is a variable.  The \funcarg{counts}
vector contains the number of proposal draws that were necessary to
accept that particular posterior draw.  This vector is needed to
compute the log marginal likelihood (LML, see below), and to assess
the efficiency of the algorithm.  The \funcarg{gt.1} vector flags whether a
threshold value was greater than 1.  This is an important check,
because if any elements of that vector are 1, that means that the
proposal density was still just a bit too tight.  If this is the case
for only a couple of draws, it's probably not a big deal.  If it is
true for a large number of draws, not only is the proposal density
invalid, but something probably went wrong earlier in the algorithm.

You can get summaries and quantiles for your parameters of interest by
applying \func{summary} or \func{quantile} (or another such function)
to each column.  In this case, let's just summarize the
population-level parameters.

\begin{knitrout}
\definecolor{shadecolor}{rgb}{0.969, 0.969, 0.969}\color{fgcolor}\begin{kframe}
\begin{alltt}
\hlstd{quants} \hlkwb{<-}  \hlstd{plyr::}\hlkwd{aaply}\hlstd{(draws[[}\hlstr{"draws"}\hlstd{]][,(N}\hlopt{*}\hlstd{k}\hlopt{+}\hlnum{1}\hlstd{)}\hlopt{:}\hlstd{nvars],} \hlnum{2}\hlstd{,}
                       \hlstd{quantile,} \hlkwc{probs}\hlstd{=}\hlkwd{c}\hlstd{(}\hlnum{.025}\hlstd{,} \hlnum{.5}\hlstd{,} \hlnum{.975}\hlstd{),}
                       \hlkwc{.parallel} \hlstd{= run.par)}
\hlstd{quants}
\end{alltt}
\begin{verbatim}
#    
# X1     2.5%     50%   97.5%
#   1 -2.4544 -1.9364 -1.3431
#   2  1.1379  1.7032  2.6554
\end{verbatim}
\end{kframe}
\end{knitrout}


If any elements in \funcarg{counts} is NA, it means that the proposal count reached the
value in \funcarg{max.tries} without an acceptance.  This would suggest an inefficient sampler that
requires further investigation.  This could be that the proposal
function is too diffuse, or, surprisingly, that it is too tight.

\section{Estimating the log marginal likelihood}


The log marginal likelihood (LML) is the likelihood of the data under
the model, $\Ly$.  For better or worse, the LML is often used for model comparison (e.g.,
computing Bayes factors).  As discussed in \citet{BraunDamien2015},
there is no generally accepted method for computing the LML from MCMC
output.  However, estimating the LML using the \func{LML} function in
this package is straightforward.  All of the arguments to \func{LML}
were used in earlier function calls, or returned by
\func{sample.GDS}.


\begin{knitrout}
\definecolor{shadecolor}{rgb}{0.969, 0.969, 0.969}\color{fgcolor}\begin{kframe}
\begin{alltt}
\hlkwa{if} \hlstd{(}\hlkwd{any}\hlstd{(}\hlkwd{is.na}\hlstd{(draws[[}\hlstr{"counts"}\hlstd{]]))) \{}
    \hlstd{LML} \hlkwb{<-} \hlnum{NA}
\hlstd{\}} \hlkwa{else} \hlstd{\{}
    \hlstd{LML} \hlkwb{<-} \hlkwd{get.LML}\hlstd{(}\hlkwc{counts}\hlstd{=draws}\hlopt{$}\hlstd{counts,}
                   \hlkwc{log.phi}\hlstd{=log.phi,}
                   \hlkwc{post.mode}\hlstd{=theta.star,}
                   \hlkwc{fn.dens.post}\hlstd{= FD}\hlopt{$}\hlstd{fn,}
                   \hlkwc{fn.dens.prop}\hlstd{=dmvn.sparse.wrap,}
                   \hlkwc{prop.params}\hlstd{=prop.params)}
\hlstd{\}}
\hlstd{draws[[}\hlstr{"LML"}\hlstd{]]} \hlkwb{<-} \hlstd{LML}
\hlstd{draws[[}\hlstr{"acc.rate"}\hlstd{]]} \hlkwb{<-} \hlnum{1}\hlopt{/}\hlkwd{mean}\hlstd{(draws}\hlopt{$}\hlstd{counts)}
\hlkwd{cat}\hlstd{(}\hlstr{"Acceptance rate: "}\hlstd{,draws[[}\hlstr{"acc.rate"}\hlstd{]],}
    \hlstr{"\textbackslash{}nLog marginal likelihood: "}\hlstd{,draws[[}\hlstr{"LML"}\hlstd{]],}\hlstr{"\textbackslash{}n"}\hlstd{)}
\end{alltt}
\begin{verbatim}
# Acceptance rate:  0.92593 
# Log marginal likelihood:  -2664.8
\end{verbatim}
\end{kframe}
\end{knitrout}


\section{Future development}

The \pkg{bayesGDS} package remains incomplete.  Here are some
improvements and enhancements that are under consideration for future releases.

\begin{enumerate}
\item A wrapper function for the entire BD algorithm;
\item Functions to summarize and display the output;
\item Options to retain a subset of variables and discard others
  (e.g., keep only population-level parameters);
\item A simpler interface for parallel sampling;
  \item Examples of parallel execution on distributed platforms;
  \item Better load balancing during parallel execution;
\item More examples of different kinds of applications; and
  \item More details in the vignette about how \func{sample.GDS} works.
\end{enumerate}



\printbibliography

\end{document}
